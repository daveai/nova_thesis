\chapter*{Abstract}
\addcontentsline{toc}{chapter}{Abstract}

Can machine learning methods outperform traditional econometric models for volatility forecasting? The answer, this thesis finds, depends on the specific asset under consideration.

This thesis compares GARCH, EGARCH, Random Forest, XGBoost, and LSTM models using daily data from 2019 to 2025 for Bitcoin, Ethereum, the S\&P 500, and VIX. Performance is measured out-of-sample via RMSE, with Diebold-Mariano tests employed to assess whether observed differences are statistically significant.

The results resist a simple ``ML wins'' or ``stick with GARCH'' conclusion. For the S\&P 500, no machine learning method significantly improves on GARCH, and LSTM performs worse than GARCH outright. The only statistically significant improvement for equities comes from EGARCH (p<0.01), which captures the leverage effect that symmetric GARCH misses. Model structure, not model complexity, is what matters for equity volatility.

For Bitcoin, all machine learning methods significantly outperform GARCH, with XGBoost achieving the lowest RMSE and an improvement of approximately 11 per cent relative to the GARCH baseline. But for Ethereum, machine learning offers no significant improvement; EGARCH performs best. Same asset class. Different outcomes.

From a practical standpoint, the 11\% RMSE reduction for Bitcoin could meaningfully affect risk management and position sizing decisions. For equity markets, EGARCH represents the appropriate upgrade from basic GARCH. For Ethereum, the additional model complexity appears to offer no benefit.

The findings suggest that model selection needs to be asset-specific and validated through statistical testing rather than RMSE rankings alone. Deep learning neither ``always works'' nor ``never works'' for volatility. It depends on the asset.

% This line shows the Keywords
\imspagekeyws
