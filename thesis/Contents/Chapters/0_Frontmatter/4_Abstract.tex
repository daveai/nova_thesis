\chapter*{Abstract}
\addcontentsline{toc}{chapter}{Abstract}

Can machine learning methods outperform traditional econometric models for volatility forecasting? The answer, this thesis finds, depends on the specific asset under consideration.

This thesis compares GARCH, EGARCH, Random Forest, XGBoost, and LSTM models using daily data from 2019 to 2025 for Bitcoin, Ethereum, the S\&P 500, and VIX. Performance is measured out-of-sample via RMSE, with Diebold-Mariano tests employed to assess whether observed differences are statistically significant.

The results prove more nuanced than a simple ``ML wins'' or ``stick with GARCH'' conclusion would suggest. For the S\&P 500, LSTM achieves the strongest performance: a 17\% RMSE improvement over GARCH that is statistically significant (p<0.01). This finding suggests that neural networks may capture patterns in equity volatility that traditional models do not.

For Bitcoin, all machine learning methods significantly outperform GARCH, with XGBoost achieving the lowest RMSE. But for Ethereum, machine learning offers no improvement; EGARCH performs best. Same asset class. Different outcomes.

From a practical standpoint, the 17\% RMSE reduction for equities could meaningfully affect risk management and position sizing decisions. For Bitcoin, the 11\% improvement from machine learning methods may similarly carry practical significance. For Ethereum, the additional model complexity appears to offer no benefit.

The findings suggest that model selection needs to be asset-specific and validated through statistical testing rather than RMSE rankings alone. Deep learning neither ``always works'' nor ``never works'' for volatility. It depends on the asset.

% This line shows the Keywords
\imspagekeyws
